\documentclass[a4paper,12pt]{article}

\usepackage[utf8]{inputenc}
\usepackage[T1]{fontenc}
\usepackage{polski}
\usepackage{amsmath}
\usepackage{amssymb}
\usepackage{geometry}
\usepackage{graphicx}
\usepackage{listings}
\usepackage{xcolor}
\usepackage{hyperref}
\usepackage{float}
\usepackage{pgfplots}

% Ustawienie kompatybilności wykresów
\pgfplotsset{compat=1.18}

% Ustawienia marginesów
\geometry{
 a4paper,
 left=25mm,
 right=25mm,
 top=25mm,
 bottom=25mm,
}

% Ustawienia kolorowania kodu (C++)
\definecolor{codegreen}{rgb}{0,0.6,0}
\definecolor{codegray}{rgb}{0.5,0.5,0.5}
\definecolor{codepurple}{rgb}{0.58,0,0.82}
\definecolor{backcolour}{rgb}{0.95,0.95,0.92}

\lstdefinestyle{mystyle}{
    backgroundcolor=\color{backcolour},   
    commentstyle=\color{codegreen},
    keywordstyle=\color{magenta},
    numberstyle=\tiny\color{codegray},
    stringstyle=\color{codepurple},
    basicstyle=\ttfamily\footnotesize,
    breakatwhitespace=false,         
    breaklines=true,                 
    captionpos=b,                    
    keepspaces=true,                 
    numbers=left,                    
    numbersep=5pt,                  
    showspaces=false,                
    showstringspaces=false,
    showtabs=false,                  
    tabsize=4
}

\lstset{style=mystyle}

\title{\textbf{Równania Różniczkowe i Różnicowe}\\Zadanie obliczeniowe MES\\Wariant 4.2: Wibracje akustyczne warstwy materiału}
\author{Jacek Łoboda}
\date{\today}

\begin{document}

\maketitle

\section{Sformułowanie problemu}

Celem zadania jest rozwiązanie równania różniczkowego metodą elementów skończonych.

Równanie różniczkowe w dziedzinie $\Omega = (0, 2)$:
\begin{equation}
    -\frac{d^2u(x)}{dx^2} - u(x) = \sin(x) \quad \text{dla } x \in (0, 2)
    \label{eq:strong_form}
\end{equation}

Warunki brzegowe:
\begin{align}
    u(0) &= 1 \quad (\text{Warunek Dirichleta}) \label{eq:bc_dirichlet} \\
    u'(2) - u(2) &= 5 \quad (\text{Warunek Robina}) \label{eq:bc_robin}
\end{align}

\section{Sformułowanie wariacyjne}

Aby zastosować metodę elementów skończonych, należy przekształcić równanie (\ref{eq:strong_form}) do postaci słabej (wariacyjnej).

Mnożymy równanie stronami przez funkcję testową $v(x) \in V$, taką że $v(0) = 0$ (ze względu na warunek Dirichleta w punkcie $x=0$), i całkujemy po dziedzinie:

\begin{equation}
    \int_0^2 \left( -u''(x) - u(x) \right) v(x) \, dx = \int_0^2 \sin(x) v(x) \, dx
\end{equation}

Korzystając z liniowości całki:
\begin{equation}
    -\int_0^2 u''(x)v(x) \, dx - \int_0^2 u(x)v(x) \, dx = \int_0^2 \sin(x)v(x) \, dx
\end{equation}

Całkujemy pierwszy człon przez części:
\begin{equation}
    \int_0^2 u''(x)v(x) \, dx = \left[ u'(x)v(x) \right]_0^2 - \int_0^2 u'(x)v'(x) \, dx
\end{equation}

Podstawiając to do równania otrzymujemy:
\begin{equation}
    -\left( u'(2)v(2) - u'(0)\underbrace{v(0)}_{0} \right) + \int_0^2 u'(x)v'(x) \, dx - \int_0^2 u(x)v(x) \, dx = \int_0^2 \sin(x)v(x) \, dx
\end{equation}

Uwzględniamy warunek brzegowy Robina (\ref{eq:bc_robin}). Przekształcamy go do postaci $u'(2) = u(2) + 5$ i podstawiamy:
\begin{equation}
    -\left( (u(2) + 5)v(2) \right) + \int_0^2 u'(x)v'(x) \, dx - \int_0^2 u(x)v(x) \, dx = \int_0^2 \sin(x)v(x) \, dx
\end{equation}

Porządkujemy równanie, przenosząc znane wartości na prawą stronę:
\begin{equation}
    \int_0^2 u'(x)v'(x) \, dx - \int_0^2 u(x)v(x) \, dx - u(2)v(2) = \int_0^2 \sin(x)v(x) \, dx + 5v(2)
\end{equation}

\subsection{Homogenizacja warunku brzegowego}
Ponieważ $u(0) = 1 \neq 0$, stosujemy podstawienie:
\begin{equation}
    u(x) = w(x) + \tilde{u}(x), \quad \text{gdzie } \tilde{u}(x) = 1
\end{equation}
Wtedy $w(0) = 0$, więc $w \in V$. Pochodna $\tilde{u}'(x) = 0$.
Forma dwuliniowa $B(u,v)$:
\begin{equation}
    B(u,v) = \int_0^2 u'v' \, dx - \int_0^2 uv \, dx - u(2)v(2)
\end{equation}
Liniowa forma funkcjonalna $L(v)$:
\begin{equation}
    L(v) = \int_0^2 \sin(x)v \, dx + 5v(2)
\end{equation}

Podstawiając $u = w + 1$:
\begin{equation}
    B(w, v) = L(v) - B(1, v)
\end{equation}
Obliczamy $B(1, v)$:
\begin{equation}
    B(1, v) = \int_0^2 (0)v' \, dx - \int_0^2 (1)v \, dx - (1)v(2) = - \int_0^2 v \, dx - v(2)
\end{equation}
Ostateczne równanie wariacyjne dla funkcji $w$:
\begin{equation}
    B(w,v) = \int_0^2 \sin(x)v \, dx + 5v(2) - \left( - \int_0^2 v \, dx - v(2) \right)
\end{equation}
\begin{equation}
    \boxed{ \int_0^2 w'v' \, dx - \int_0^2 wv \, dx - w(2)v(2) = \int_0^2 (\sin(x) + 1)v \, dx + 6v(2) }
\end{equation}

\section{Metodyka rozwiązania}

Problem rozwiązano przy użyciu:
\begin{itemize}
    \item Dyskretyzacji odcinka $[0, 2]$ na $n$ elementów skończonych.
    \item Liniowych funkcji kształtu (baza Lagrange'a).
    \item Numerycznego całkowania metodą kwadratury Gaussa-Legendre'a (2 punkty).
    \item Algorytmu Thomasa do rozwiązania trójdiagonalnego układu równań liniowych.
\end{itemize}

Ostateczne rozwiązanie uzyskuje się poprzez dodanie przesunięcia: $u_h(x) = w_h(x) + 1$.

\section{Wyniki obliczeń}

Przeprowadzono symulację dla liczby elementów $n = 100$.

\textbf{Sprawdzenie warunków brzegowych:}
\begin{itemize}
    \item Wartość w $x=0$: $u(0) = 1.000000$ (Zgodne z założeniem).
    \item Wartość w $x=2$: $u(2) \approx -5.0819$.
    \item Pochodna numeryczna w $x=2$: $u'(2) \approx -0.1237$.
    \item Sprawdzenie warunku Robina: $u'(2) - u(2) \approx -0.1237 - (-5.0819) = 4.9582$.
\end{itemize}
Otrzymany wynik jest bardzo bliski wartości teoretycznej $5$, co potwierdza poprawność implementacji.

\begin{figure}[H]
    \centering
    
    \begin{tikzpicture}
        \begin{axis}[
            width=12cm, height=8cm,
            title={Wykres rozwiązania przybliżonego $u(x)$},
            xlabel={$x$},
            ylabel={$u(x)$},
            grid=major,
            legend pos=south west,
            cycle list name=color list
        ]
            
            \addplot[
                color=blue,
                mark=none,
                very thick
            ]
            table[
                col sep=semicolon,  % Separator średnik
                x index=0,          % Pierwsza kolumna
                y index=1,          % Druga kolumna
                header=true         % Jest nagłówek
            ] {results.csv};        % Nazwa pliku
            
            \addlegendentry{Rozwiązanie MES}
            
        \end{axis}
    \end{tikzpicture}
    % Włączamy z powrotem, jeśli potrzebne (zazwyczaj nie trzeba na końcu figury)
    % \shorthandon{;} 
    \caption{Wykres wygenerowany na podstawie danych z pliku CSV.}
    \label{fig:wykres_pgf}
\end{figure}

\end{document}